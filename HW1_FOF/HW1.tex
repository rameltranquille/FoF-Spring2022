%=======================02-713 LaTeX template, following the 15-210 template==================
%
% You don't need to use LaTeX or this template, but you must turn your homework in as
% a typeset PDF somehow.
%
% How to use:
%    1. Update your information in section "A" below
%    2. Write your answers in section "B" below. Precede answers for all 
%       parts of a question with the command "\question{n}{desc}" where n is
%       the question number and "desc" is a short, one-line description of 
%       the problem. There is no need to restate the problem.
%    3. If a question has multiple parts, precede the answer to part x with the
%       command "\part{x}".
%    4. If a problem asks you to design an algorithm, use the commands
%       \algorithm, \correctness, \runtime to precede your discussion of the 
%       description of the algorithm, its correctness, and its running time, respectively.
%    5. You can include graphics by using the command \includegraphics{FILENAME}
%
\documentclass[11pt]{article}
\usepackage{amsmath,amssymb,amsthm}
\usepackage{graphicx}
\usepackage[margin=1in]{geometry}
\usepackage{fancyhdr}
\setlength{\parindent}{0pt}
\setlength{\parskip}{5pt plus 1pt}
\setlength{\headheight}{13.6pt}
\newcommand\question[2]{\vspace{.25in}\hrule\textbf{#1: } #2\vspace{.5em}\hrule\vspace{.10in}}
\renewcommand\part[1]{\vspace{.10in}\textbf{(#1)}}
\newcommand\runtime{\vspace{.10in}\textbf{Running time: }}
\pagestyle{fancyplain}
\lhead{\textbf{\NAME\ (\ANDREWID)}}
\chead{\textbf{Foundations of Finance HW\HWNUM}}
\rhead{02-713, \today}
\begin{document}\raggedright
%Section A==============Change the values below to match your information==================
\newcommand\NAME{Ramel Tranquille}  % your name
\newcommand\ANDREWID{rt1734}     % your andrew id
\newcommand\HWNUM{1}              % the homework number
%Section B==============Put your answers to the questions below here=======================

% no need to restate the problem --- the graders know which problem is which,
% but replacing "The First Problem" with a short phrase will help you remember
% which problem this is when you read over your homeworks to study.

\section{Financial Markets}
The starting bid price $\$= 102.25$ with ask of $\$102.50$ and zero inventory
\question{1}{Day One} 
On day 1, there we recieve orders of
buy - 10,000 shares 
sell - 4,000 shares 

On the 4000 shares bought and sold,
\[
    4000 * 102.25 = 409000 \text{ and } 4000 * 102.50 = 410000
\]   
Thus, $\$410000 - \$409000 = \$1000$ with the remaining being $-6000$ shares since we have more buy orders. \\
We can use the value in between ask/bid prices, being 102.375. Thus, inventory at day-end is 
\[
    I = -6000 * \frac{102.5-102.25}{2} = -6000 * 102.375 = -\$614250
\]
You can also use other ways to value inventory such as only using either the bid or ask price. With the bid price, 
inventory is $-6000 * 102.25 = -\$613500$ and in contrast inventory with the ask price $-6000 * 102.5 = -\$615000$.

\question{1}{Day Two}
Prices change jump to $110.25=$bid price and $110.5=$ask price. And we have -6000 shares of inventory. On the 2nd day we recieve, 
market buy order of 2,000 shares such that $2000 * 110.50 = \$221000$ and market sell order of 8,000 shares such that $8000 * 110.25 = \$-882000$\\

Using the ask price (since they were sold from us) and shares from day one, $-6000 * 102.5 = -\$615000$.
\[
    \text{total-profit} = \text{day2}(221000 - 882000) + \text{day1}(615000) = - 46000
\]
Since, the inventory of day one is -6000 and day two is $8000-2000=6000$, thus invenotry is 0. 

\question{1}{Market Makers/Dealers}
The price of the market maker's (dealers) services is the bid-ask spread. Thus, they take risk in inventory instead of 
predicting prices/speculating. Yes we could have stopped taking (or not filled) buy orders from day one, such that 
we are not short -6000 in inventory. This action would leave us with \$1000 profit for the day and improve 
overall performance for day one and two. This action is consistent with market makers. 

\section{Return Measures}
$return/day = 1\% = r(t)$ with 250 trading days per year
\question{2}{Geometric Average} 
EAR (effective annual rate) - percentage increase in funds invested over $T$ horizon (1 year - 250 days)
$$ 
(1 + r(t))^{t} = [1+EAR] \rightarrow
(1 + .01)^{250} - 1 = 11.0321557 = EAR
$$ 
$$ 
(1 + r(t))^{t} = [1+EAR] \rightarrow
(1 + .01)^{250} - 1 = 11.0321557 = EAR = 1103.216\%
$$ 
And $(EAR*100) + 100$ is the final return + initial investment after a year. \\ 
The EAR is the geometric average return $1.01^{250} - 1 = 11.032$


\question{2}{Simple/Arithmetic Average}
With a $1\%$ return rate and $0\%$ interest rate from the checking account (= no reinvesting). 
$R_n = R_1+R_2+R_3+\dots+R_T$ with $R_i$ (the daily return) being $\$1$ and $1*250$  earning after the end of a year with 250 trading days.

\question{2}{Arithmetic vs. Geometric}
arithmetic - It is proper to use simple when there is no-reinvesting or interest is not applied, such as 2B

geometric - It is proper to use geometric average when we can rinvest the earnings at a periodic rate (e.g. 1 percents per trading day)
since geometric allows for compounding

\question{3}{Alternatives} 
i. 8\% compounded annually 
$$
((1 + .08))^{1}-1) = EAR = .08
$$
ii. 8\% compounded daily
$$
((1 + .08))^{250}-1) = EAR_{daily} = 226954537.4\%
$$
\question{4}{\textit{Excel Question}} 
For my data, the start date is 1/1/1988 with price $\$120.43$ and the final date is 02/18/22 with $P_1 = 2208.78$
\[
    HPR = \frac{P_1 + C}{P_0} - 1 = \frac{2208.78 + 0}{120.43} = 18.34077
\]
Thus the we can compute the annualized HPR as 
\[
    HPR_{ann} = (1+HPR)^{1/T} - 1 = 19.34077^{1/34} - 1 = 1.091 = 0.1091\%
\]

\section{Time Value of Money}
\question{5}{Present-Value of Bonds} 
Determing the present-value price of a 5yr-bond with $\$1000$ face value and $\$5\%$ yield
\[
   present-value = \frac{future-value}{(1+yield)^T} = \frac{1000}{(1+0.05)^5} = \frac{1000}{1.27628} = 783.5262
\]

\question{5}{Maturity} 
Now, price $=\$325.57$ thus the maturity of the bond is 
\[
    future-value = price(1+yield)^T \rightarrow 1000 = 325.57(1.05)^T \rightarrow 3.071536 = 1.05^T
\]
Solving for T
\[
   Log(3.07154) = Log(1.05) * T \rightarrow 0.487355 = 0.0211892991 * T 
\]
Thus, 
\[
    T = \frac{0.487355}{0.0211892991} = 23.00008217
\]
Maturity for the 5-year bond at price \$325.57 is 23 years.

\question{6}{Investment Choice} 
If you were not to reinvesting your annual returns (or removing money from your account), option $A$ is better. Otherwise the we can use HPR
$$ 
(1 + r(t))^{t} = [1+EAR] \rightarrow (1.55)^{10} - 1= EAR = 79.0418249
$$ 
Then, 
\[
    79.0418249 * \$550 = 43473.0037
\]
Since, the returns from the investment into Chase is greater than that of a 10-yr bond and the interest is 
guaranteed, I would prefer option $A$.


\question{7}{Annuities and Perpetuities} 
For the present-value of annuity, with periodic-payment $=10,000$,rate $=0.05$, and period $=6$
\[
    P * (\frac{1-\frac{1}{(1+r)^{n}}}{r}) = present-value
\]
\[
    10000 * (\frac{1-\frac{1}{(1.05)^{6}}}{.05}) = 50,756.92068
\]
Next the present-value of the 6-year annuity at rate=$0.10$. 
\[
    10000 * (\frac{1-\frac{1}{(1.1)^{6}}}{.1}) = 43,552.61
\]
For the present-value of perpetuity, with periodic-payment $=10,000$ and rate $=0.05$,
\[
    PV = C / R = 10000 / 0.05 =  200000
\]
Since this is the value of the perpetuity after 10 years, we work backwards, to find today's present value,
\[
    FV / (1+rate)^{periods} = present-value = 200,000/(1.05)^10 = 200,000/1.62889 = 122782.6507
\]

Finally, the present-value of the perpetuity at rate $=.1$.
\[
    PV = C / R = 10,000 / 0.1 = 100,000
\]
and the present-value of the perpetuity at rate $=.1$ today, (instead of 10 years from now) is
\[
    PV = \frac{FV}{(1+r)^n} = 100,000 / 1.1^{10} = 100,000 / 2.593742 = 38554.33
\]
Thus, after all the computation, it seems that the perpetuity of 5\% should be chosen as it has the higher present-value. 
However, the annuity should be chosen at the rate, 10\% for the same reason.

\end{document}
